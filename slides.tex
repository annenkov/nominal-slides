\documentclass[10pt]{beamer}
\setbeamertemplate{navigation symbols}{}
\usetheme{Copenhagen}
\usepackage{xcolor}
\usepackage{pifont}
\usepackage{listings}
\usepackage{todonotes}
\lstloadlanguages{Haskell,Java}

\usepackage{amsmath, listings, amsthm, xspace, semantic, stmaryrd, mathrsfs}
\usepackage{mathtools}
\usepackage[utf8x]{inputenc}
\usepackage{amssymb}
\usepackage{subcaption}
\usepackage[all]{xy}
\usepackage{amsbsy}
\usepackage[threshold=1]{csquotes}

\usepackage{listings}
\usepackage{lstcoq}
\usepackage{mathpartir}

\usepackage{xcolor}
% colors for Coq syntax highlighting
\definecolor{ltblue}{rgb}{0,0.4,0.4}
\definecolor{dkblue}{rgb}{0.1, 0.2, 0.6}
\definecolor{dkgreen}{rgb}{0.1, 0.5, 0.1}
\definecolor{dkviolet}{rgb}{0.3,0,0.5}
\definecolor{dkred}{rgb}{0.5,0,0}
\definecolor{redd}{rgb}{0.7, 0.1, 0.1}

%% Lean code colors
\definecolor{keywordcolor}{rgb}{0.7, 0.1, 0.1}   % red
\definecolor{tacticcolor}{rgb}{0.1, 0.2, 0.6}    % blue
\definecolor{commentcolor}{rgb}{0.4, 0.4, 0.4}   % grey
\definecolor{symbolcolor}{rgb}{0.0, 0.1, 0.6}    % blue
\definecolor{sortcolor}{rgb}{0.1, 0.5, 0.1}      % green

\def\lstlanguagefiles{lstlean.tex}

\usepackage[english]{babel}

% definitions for the first chapter, notation from the NWPT'16 abstract
\newcommand\tuple[1]{\langle#1\rangle}

\newcommand\ilSem[2]{{\mathcal {IL}}\left\llbracket#1\right\rrbracket_{#2}}
\newcommand\sem[1]{\left\llbracket#1\right\rrbracket}
\newcommand\tSem[2]{{\mathcal {T}}\left\llbracket#1\right\rrbracket_{#2}}
\newcommand\cSem[2]{{\mathcal C}\left\llbracket#1\right\rrbracket_{#2}}
\newcommand\gSem[1]{\left\llbracket#1\right\rrbracket}
\newcommand\tysem[1]{\left\llbracket#1\right\rrbracket}
\newcommand\compileExp[2] {{\tau_{\textrm{e}}}\left\llbracket#1\right\rrbracket_{#2}}
\newcommand\compileContr[2]{{\tau_{\textrm{c}}}\left\llbracket#1\right\rrbracket_{#2}}
\newcommand\compileOp[2] {{\tau_{\textrm{op}}}\left\llbracket#1\right\rrbracket_{#2}}
\newcommand\eSem[2]{{\mathcal E}\left\llbracket#1\right\rrbracket_{#2}}
\newcommand\cRed[2]{\stackrel{#2}{\Longrightarrow_{#1}}}
\newcommand\opsem[1]{\left\llbracket#1\right\rrbracket}

\newcommand\type[1]{\mathsf{#1}}
\newcommand\obs{\mathbf{obs}}
\newcommand\acc{\mathbf{acc}}
\newcommand\advEnv[2]{#1/#2}

\newcommand\cutPayoff[1]{\mathsf{cutPayoff}(#1)}
\newcommand\zero{\emptyset}
\newcommand\scale{\times}
\newcommand\both{\mathop{\&}}
\newcommand\transl{\uparrow}
\newcommand\promote{\Uparrow}
\newcommand\transfer[3]{{#1(#2 \to #3)}}
\newcommand\letc[3]{\letA\;#1 = #2\;\letB\;#3}
\newcommand\letA{\mathbf{let}}
\newcommand\letB{\mathbf{in}}
\newcommand\ifthen[3]{\mathtt{if} \;#1\;  \mathtt{then} \;#2\;
\mathtt{else} \; #3}
\newcommand\ifwithin[4]{
\mathtt{if} \;#1\; \mathtt{within} \;#2\; \mathtt{then} \;#3\;
\mathtt{else} \; #4}
\newcommand\pto{\rightharpoonup}

\newcommand\id[1]{\ensuremath{\mathit{#1}}}

\reservestyle{\command}{\mathtt}
\command{if[if],model[model],not[not],neg[neg],loopif[loopif],
  payoff[payoff],add[add],sub[sub],mult[mult],div[div],lt[lt],and[and],true[true],
  or[or],scale[scale],translate[translate],zero[zero],
  transfer[transfer],both[both],obs[obs],now[now],let[let],Tnum[Tnum],Tvar[Tvar],
  checkWithin[checkWithin],rNum[rNum],bNum[bNum],Tplus[Tplus],Texpr[Texpr],TnumZ[TnumZ],cond[cond],tplus[tplus]}

\newcommand{\kt}[1]{\texttt{#1}}

%% \theoremstyle{plain}
%% \newtheorem{thm}{Theorem}[section]
%% \newtheorem{conj}{Conjecture}[section]
%% \newtheorem{lemma}{Lemma}[section]

%% \theoremstyle{definition}
%% \newtheorem{defn}{Definition}[section]
%% \newtheorem{remark}{Remark}[section]
%% \newtheoremstyle{case}{}{}{}{}{}{:}{ }{}
%% \theoremstyle{case}
%% \newtheorem{case}{Case}
%% \makeatletter
%% \@addtoreset{case}{theorem}
%% \@addtoreset{case}{lemma}
%% \makeatother
%% \newtheorem{example}{Example}[chapter]

%% \renewcommand\qedsymbol{$\blacksquare$}


% inlined source code
\newcommand{\icode}[1]{\lstinline[mathescape]!#1!}
% inlined source code in theorems
\newcommand{\icodet}[1]{\textup{\lstinline{#1}}}

%% defs for the chapter 2
\newcommand{\Atom}{\mathbb{A}}
\newcommand{\Perm}{\id{Perm}~\Atom}
\newcommand{\SubPerm}[1]{\id{Perm}_{#1}~\Atom}
\newcommand{\swap}[2]{(#1 ~ #2)}
\renewcommand{\action}[2]{#1 \cdot #2}
\newcommand{\aaction}[3]{#2 \cdot_{#1} #3}
\newcommand{\supp}[1]{\id{supp}~ #1}
\newcommand{\asupp}[2]{\id{supp}_{#1}~#2}
\newcommand{\card}[1]{\id{card}~#1}
\newcommand{\Nom}{\mathbf{Nom}}

\newcommand{\finmap}{\stackrel{\textrm{fin}}{\rightarrow}}
\newcommand{\TE}{\id{TE}\,}
\newcommand{\VE}{\id{VE}\,}
\newcommand{\ME}{\id{ME}\,}
\newcommand{\MTE}{\id{MTE}\,}
\newcommand{\SE}{\id{SE}\,}
\newcommand{\Env}{{\textrm{Env}}}
\newcommand{\TEnv}{{\textrm{TEnv}}}
\newcommand{\VEnv}{{\textrm{VEnv}}}
\newcommand{\MEnv}{{\textrm{MEnv}}}
\newcommand{\MTEnv}{{\textrm{MTEnv}}}
\newcommand{\Mid}{{\textrm{Mid}}}
\newcommand{\MTid}{{\textrm{MTid}}}
\newcommand{\Mod}{{\textrm{Mod}}}
\newcommand{\FunSig}{{\textrm{FunSig}}}
\newcommand{\TSet}{{\textrm{TSet}}}
\newcommand{\LSet}{{\textrm{LSet}}}
\newcommand{\NSet}{{\textrm{NSet}}}
\newcommand{\Finset}{{\textrm{Fin}}}
\newcommand{\MTy}{{\textrm{MTy}}}
\newcommand{\kw}[1]{\mbox{\bfseries{#1}}}
\newcommand{\sembox}[1]{\hfill \normalfont \mbox{\fbox{\(#1\)}}}
\newcommand{\titlesembox}[2]{\subsubsection*{#1}\sembox{#2}}
\newcommand{\Dom}{\textrm{Dom}}

\newcommand{\hblue}[1]{\textcolor{blue}{#1}}
\newcommand{\hgreen}[1]{\textcolor{sortcolor}{#1}}
\newcommand{\hviolet}[1]{\textcolor{dkviolet}{#1}}
\newcommand{\hred}[1]{\textcolor{red}{#1}}

\lstset{language=coq}

\title{Nominal Techniques in Coq}

\author{Danil Annenkov}
\institute{University of Copenhagen}
\date{November 16, 2017}

\begin{document}
\maketitle

\AtBeginSection[]
{
   \begin{frame}
       \frametitle{Outline}
       \tableofcontents[currentsection]
   \end{frame}
}

\begin{frame}
  \frametitle{Background}

  \only<1> {\begin{exampleblock}{}
    \large{There are only two hard things in Computer Science: cache
    invalidation and naming things.}

    \vskip5mm
  \hspace*\fill{\small --- Phil Karlton}
  \end{exampleblock}}

  \only<2> {\begin{exampleblock}{}
    \large{There are only two hard things in Computer Science: cache
      invalidation and \textbf{naming things}.}

    \vskip5mm
    \hspace*\fill{\small --- Phil Karlton}
  \end{exampleblock}

  The other side of naming things is to be independent of names!}

\end{frame}

\begin{frame}
  \frametitle{Variable binding}
  \begin{itemize}
  \item Variable binding is a ubiquitous concept in the programming language research.
  \item One wants definitions to be independent of the choice of names
    for bound variables.
  \item It is relatively easy to deal with binding in pen-and-paper proofs.
  \item It is notoriously hard to deal with in proof assistants.
  \end{itemize}
\end{frame}

\begin{frame}[fragile]
  \frametitle{Variable binding: Examples}
  \begin{table}
    \begin{tabular}{p{5cm} p{5cm}}
      \hline
      \multicolumn{1}{c}{\bfseries Haskell} & \multicolumn{1}{c}{\bfseries Java}\\
      \hline
      \begin{lstlisting}[language=Haskell]
plusTwo a = let b = 2
            $~$ in  a + b
      \end{lstlisting}&
      \begin{lstlisting}[language=Java]
public int plusTwo (int a) {
    int b = 2;
    return a + b; }
      \end{lstlisting}
    \end{tabular}
  \end{table}\vspace{-1.5em}
  \pause
  \centering
  We can pick other names for \icode{a} and \icode{b}:
  \begin{table}\vspace{-2.5em}
  \begin{tabular}{p{5cm} p{5cm}}
      \begin{lstlisting}[language=Haskell]
plusTwo c = let d = 2
            $~$ in c + d
      \end{lstlisting}&
      \begin{lstlisting}[language=Java]
public int plusTwo (int c) {
    int d = 2;
    return c + d; }
      \end{lstlisting}
  \end{tabular}
  \end{table}\vspace{-1.5em}
  \pause
  But not arbitrary names: \textbf{variable capture}!
  \begin{table}\vspace{-2.5em}
  \begin{tabular}{p{5cm} p{5cm}}
  \begin{lstlisting}[language=Haskell]
plusTwo b = let b = 2
            $~$ in b + b
      \end{lstlisting}&
      \begin{lstlisting}[language=Java]
public int plusTwo (int b) {
    int b = 2;
    return b + b; }
      \end{lstlisting}
  \end{tabular}
  \end{table}
\end{frame}

\begin{frame}
  \frametitle{Simply-Typed Lambda Calculus}
  \begin{itemize}
  \item From now on we will switch to the Simply-Typed Lambda Calculus (STLC).
  \item STLC is well-studied and has a simple binding structure.
  \item The grammar of (raw) lambda terms:\\
    $e \in \texttt{Lam}  ::= v ~|~ \lambda x.e ~|~ e_1 e_2$
  \end{itemize}
\end{frame}

\begin{frame}
  \frametitle{Variable Convention}
  \begin{exampleblock}{Barendregt's Variable Convention}
     \large{If $M_1,\dots, M_n$ occur in a certain mathematical context
     (e.g. definition, proof), then in these terms all bound
     variables are chosen to be different from the free variables.}
  \end{exampleblock}
\end{frame}

\begin{frame}
  \frametitle{Variable Convention}
  \begin{itemize}
  \item Consider the following typing rule for lambda abstraction:
    \[ \inferrule{\Gamma, x:\tau_1 \vdash e : \tau_2 \\ x \notin \id{dom}(\Gamma)}
              {\Gamma \vdash \lambda x.e : \tau_1 -> \tau_2} \]
  \item In order prove weakening
    \[\forall \Gamma, \Gamma', e, \tau,~~\Gamma |- e : \tau
    \land \Gamma \subseteq \Gamma' => \Gamma' |- e : \tau\]
  in case of lambda abstraction one have to show
  $x \notin \id{dom}(\Gamma')$, knowing only $x \notin \id{dom}(\Gamma)$.
\item This is possible using the variable convention.
  \end{itemize}
  \pause
  \centering
  \large{\textcolor{red}{\ding{55}}  Not enough to formalise in a proof assistant!}
\end{frame}

\begin{frame}
  \frametitle{$\alpha$-equivalence}
  \begin{itemize}
  \item We want to identify lambda-expressions up to renaming
    of bound variables.
  \item For example, $\lambda x.x =_\alpha \lambda y.y$.
  \item We could use substitution. Since $y[x/y] = x$, we say that
    $\lambda x.x =_\alpha \lambda y.y$.
  \item But substitution must be capture-avoiding, otherwise we would identify
    $\lambda y.y~x$ and $\lambda x.x~x$.
  \end{itemize}
\end{frame}

\begin{frame}
  \frametitle{$\alpha$-conversion with transpositions}
  \begin{itemize}
  \item A \emph{transposition} swaps two names:
    \[ \swap{a}{b}~c =
    \begin{cases}
      a, \text{if } b = c\\
      b, \text{if } a = c\\
      c, \text{otherwise}
    \end{cases} \]
  \item We apply transpositions to \textbf{all} occurrences of
    variables in the lambda-expression:
    $\action{\swap{y}{x}}{(\lambda y.y)} = \lambda x.x$
  \end{itemize}
\end{frame}

\begin{frame}
  \frametitle{$\alpha$-conversion with transpositions}
  Important differences with substitution-based definitions:
  \begin{itemize}
  \item Transpositions cannot lead to variable capture:
    $\action{\swap{y}{x}}{(\lambda y.y~x)} = \lambda x.x~y$
  \item We can implement the capture-avoiding substitution behavior
    using restrictions on variables (we write $x \# y$ for $x \neq y$
    and say ``$x$ is fresh for ``y''):\\
    \smallskip
    pick $z \# x$ and $z \# y$, then
    $\action{\swap{z}{y}}{(\lambda y.y~x)} = \lambda z.z~x$
  \end{itemize}
\end{frame}

\begin{frame}
  \frametitle{Nominal Sets}
  \begin{itemize}
  \item A transposition is special case of a \emph{finitary} permutation: a bijective function
    on atoms, which is not the identity on a finite subset of atoms.
    \pause
  \item The theory of nominal sets [Gabbay and Pitts 1999, Pitts 2013] is a
    \begin{quote}
    mathematical theory of names: scope, binding, freshness.
    \end{quote}
    \pause
    \item Uniform theory based on notions of permutation of variables and finite support.
    \item Applies to various binding structures.
    \item Allows to bring formalisations in a proof assistant closer to pen-and-paper proofs.
    \end{itemize}
\end{frame}

\begin{frame}
  \frametitle{Nominal Sets : Definitions}
  \begin{itemize}
  \item Assume a countably infinite set $\{a,b,c,\dots \}$ of \emph{atoms} $\Atom$:
    \[ AtomInf : \forall X \in \mathcal{P}_{fin}(\Atom),~\exists a,~a \notin X\]
  \item The \emph{action} of a permutation $\pi$ on a set $X$ is an operation
    $\action \pi -:  X -> X$ with the following properties:
    \begin{itemize}
    \item for any $x\in X$, $\action{\id{id}} x = x$
    \item for any $x\in X$, permutations $\pi_1$ and $\pi_2$, $\action{\pi_1}{(\action{\pi_2}{x})} = \action{(\pi_1 \circ \pi_2)}x$
    \end{itemize}
  \item Finite support in terms of transpositions:
    \[ \forall a,b \notin \supp x.~\action{\swap{a}{b}} x = x \]
  \item A \emph{nominal set} \textbf{X} is a set $X$, equipped with the action
    $\action \pi -$, s.t. each element in $X$ is finitely supported.
  \end{itemize}
\end{frame}

\begin{frame}
  \frametitle{Nominal Set of Lambda Expressions}
  \begin{itemize}
  \item Action (a permutation is applied to \textbf{all} occurrences of atoms uniformly):
    \begin{align*}
    \action \pi v &= \pi v\\
    \action \pi {(\lambda x.e)} &= \lambda (\pi x). \action \pi e\\
    \action \pi {(e_1 e_2)} &= (\action \pi {e_1}) (\action \pi {e_2})
    \end{align*}
  \item Support is a set of \textbf{all} atoms:
    \begin{align*}
    \supp v &= \{v\}\\
    \supp {(\lambda x.e)} &= \{x\} \cup \supp e\\
    \supp (e_1 e_2) & = (\supp {e_1}) \cup (\supp {e_2})
  \end{align*}
  \end{itemize}
\end{frame}

\begin{frame}
  \frametitle{Nominal Techniques: $\alpha$-equivalence}
  We can define $\alpha$-equivalence just in terms of the freshness relation
  and transpositions:
  \vspace{-2em}
  \begin{mathpar}
    \inferrule{\empty}
              {a =_\alpha a}

   \and

    \inferrule{t_1 =_\alpha t_1' \\
               t_2 =_\alpha t_2'}
              {t_1 t_2 =_\alpha t_1' t_2'}
   \and

   \inferrule{\action{\swap{a_1}{b}}{t_1} =_\alpha \action{\swap{a_2}{b}}{t_2} \\
               b \# (a_1,a_2,fv(t_1),fv(t_2))}
              {\lambda a_1 . t_1 =_\alpha \lambda a_2 . t_2}
  \end{mathpar}
  The freshness relation ($a$ is fresh for $x$)
  \[ a \# x = a \notin \supp x\]
  We write $a \# (x_1,\dots,x_n)$ for $a \# x_1 \land \dots \land a \# x_n$.

  \bigskip
  The support $t \in Lam/=_\alpha$ is a set of free variables of $t$.
\end{frame}

\begin{frame}
  \frametitle{Nominal Techniques: $\alpha$-equivalence}
  \only<1>{\begin{mathpar}
      \inferrule{ }
                {\lambda x.x =_\alpha \lambda y.y}
  \end{mathpar}}
  \onslide<2->{We get $z$ from the $AtomInf$ axiom with $\{x,y\}$}
  \only<2>{\begin{mathpar}
      \inferrule{z \# (x,y,\{x\},\{y\})}
                {\lambda x.x =_\alpha \lambda y.y}
  \end{mathpar}}
  \only<3>{\begin{mathpar}
      \inferrule{\action{\swap{x}{z}}{x}=_\alpha \action{\swap{y}{z}}{y} \\ z \# (x,y,\{x\},\{y\})}
                {\lambda x.x =_\alpha \lambda y.y}
  \end{mathpar}}
  \only<4>{\begin{mathpar}
      \inferrule{\inferrule{ }{z=_\alpha z} \\ z \# (x,y,\{x\},\{y\})}
                {\lambda x.x =_\alpha \lambda y.y}
    \end{mathpar}}
\end{frame}

\begin{frame}[fragile]
  \frametitle{Nominal Techniques in Coq: Permutations}
  Two ways of defining a permutation:
  \begin{itemize}
    \item
    \begin{lstlisting}
  Record Perm :=
      { perm : Atom -> Atom;
        is_biject_perm :  (is_inj perm) /\ (is_surj perm);
        has_fin_supp_perm : has_fin_supp perm}.
    \end{lstlisting}
    \pause
    We cannot use \icode{is_biject_perm} to construct an inverse permutation!
    \pause
  \item
  \begin{lstlisting}
  Record Perm :=
      { perm : Atom -> Atom;
        perm_inv : Atom -> Atom;
        l_inv : (perm_inv ∘ perm) = id ;
        r_inv : (perm ∘ perm_inv) = id ;
        fin_supp : has_fin_supp perm}.
  \end{lstlisting}
  \end{itemize}
\end{frame}

\begin{frame}[fragile]
  \frametitle{Nominal Techniques in Coq: Nominal Sets}
  We use type classes to define nominal sets:
    \begin{lstlisting}
Class NomSet :=
  { Carrier : Type;
    action : Perm -> Carrier -> Carrier;
    supp : Carrier -> FinSetA;
    action_id : forall (x : Carrier), action id_perm x = x;
    action_compose : forall (x : Carrier) (p p' : Perm),
        action p (action p' x) = action (p ∘p p') x;
    support_spec : forall  (p : Perm)  (x : Carrier),
        (forall (a : Atom), V.In a (supp x) -> p a = a) ->
        action p x = x}.
    \end{lstlisting}
\end{frame}

\begin{frame}[fragile]
  \frametitle{Nominal Techniques in Coq: Lambda Expressions}
  The nominal set of lambda expressions is an instance of \icode{NomSet}
    \begin{lstlisting}
Instance NomExp : NomSet :=
        {| Carrier := Exp;
           action := fun p e => ac_exp p e;
           supp := fun e => supp_exp e;
           action_id := fun e => (* omitted *);
           action_compose := fun e p1 p2 => (* omitted *);
           support_spec := (* omitted *) |}.
    \end{lstlisting}

\icode{ac_exp p e} recursively applies \icode{p} to all atoms in \icode{e}.

\icode{supp_exp e} returns a set of all atoms in \icode{e}.
\end{frame}

\begin{frame}[fragile]
  \frametitle{Nominal Techniques in Coq: $\alpha$-equivalence}
  The definition of $\alpha$-equivalence:
  \begin{lstlisting}
Inductive ae_exp : NomExp -> NomExp -> Prop :=
| ae_var : forall (a : NomAtom),
    (Var a) =$\alpha$ (Var a)
| ae_lam : forall (a b c : NomAtom) (e1 e2 : NomExp),
    c $\#$ (a, b, fv_exp e1, fv_exp e2) ->
    ((swap a c) @ e1) =$\alpha$ ((swap b c) @ e2) ->
    (Lam a e1) =$\alpha$ (Lam b e2)
| ae_app : forall (e1 e2 e1' e2' : NomExp),
    e1 =$\alpha$ e1' ->
    e2 =$\alpha$ e2' ->
    (App e1 e2) =$\alpha$ (App e1' e2')
where "e1 =$\alpha$ e2" := (ae_exp e1 e2).
  \end{lstlisting}

  We use the notation \icode{(swap a b) @ e} for $\action{\swap{a}{b}}{e}$.
\end{frame}

\begin{frame}
  \frametitle{Nominal Techniques: Summary}
  \begin{itemize}
  \item Actions are ``uniform'': permutations apply to
    variables in any positions - binding, bound, free.
    \pause
  \item Permutations cannot lead to variable capture.
    \pause
  \item There is a simple characterisation of $\alpha$-equivalence in
    terms of transpositions and freshness.
    \pause
  \item $\alpha$-equivalence can be generalized to various structures
    involving bound variables.
    \pause
  \item Our development is available on GitHub:
    \url{https://github.com/annenkov/stlcnorm}
  \end{itemize}
\end{frame}

\begin{frame}
  \frametitle{Nominal Techniques in Coq: Future Work}
  \begin{itemize}
  \item Ideally, we would like to have a ``nominal'' induction principle.
  \item This requires quotienting with $\alpha$-equivalence.
  \item Defining quotients in Coq is not easy.
  \item Implementation of Aydemir et al. axiomatises a nominal
    induction principle for lambda expressions quotiented with
    $\alpha$-equivalence and provides the soundness proof.
  \item Higher inductive types could be an interesting option.
  \end{itemize}
\end{frame}

\begin{frame}
  \frametitle{Nominal Techniques: Related Work}
  \begin{itemize}
  \item The most developed library for nominal techniques is the Nominal
    Isabelle package (Isabelle/HOL proof assistant) [Urban and Tasson 2005].
  \item The theory of nominal sets in Agda [Choudhury 2015].
  \item Aydemir, Bohannon, Weirich. Nominal Reasoning Techniques in Coq. 2007.

    \small{\url{http://www.seas.upenn.edu/~sweirich/papers/nominal-coq/}}
  \item Nominal techniques in Coq are part of the DeepSpec summer school course:

    \small\url{https://github.com/DeepSpec/dsss17/tree/master/Stlc}
  \end{itemize}
\end{frame}

\begin{frame}
  \frametitle{Thank you!}
  \Large Thank you for your attention!
\end{frame}
\end{document}


\end{document}
