\documentclass[10pt]{beamer}
\setbeamertemplate{navigation symbols}{}
\usetheme{Copenhagen}
\usepackage{xcolor}
\usepackage{pifont}
\usepackage{listings}
\usepackage{todonotes}

\usepackage{amsmath, listings, amsthm, xspace, semantic, stmaryrd, mathrsfs}
\usepackage{mathtools}
\usepackage[utf8x]{inputenc}
\usepackage{amssymb}
\usepackage{subcaption}
\usepackage[all]{xy}
\usepackage{amsbsy}
\usepackage[threshold=1]{csquotes}

\usepackage{listings}
\usepackage{lstcoq}
\usepackage{mathpartir}

\usepackage{xcolor}
% colors for Coq syntax highlighting
\definecolor{ltblue}{rgb}{0,0.4,0.4}
\definecolor{dkblue}{rgb}{0.1, 0.2, 0.6}
\definecolor{dkgreen}{rgb}{0.1, 0.5, 0.1}
\definecolor{dkviolet}{rgb}{0.3,0,0.5}
\definecolor{dkred}{rgb}{0.5,0,0}
\definecolor{redd}{rgb}{0.7, 0.1, 0.1}

%% Lean code colors
\definecolor{keywordcolor}{rgb}{0.7, 0.1, 0.1}   % red
\definecolor{tacticcolor}{rgb}{0.1, 0.2, 0.6}    % blue
\definecolor{commentcolor}{rgb}{0.4, 0.4, 0.4}   % grey
\definecolor{symbolcolor}{rgb}{0.0, 0.1, 0.6}    % blue
\definecolor{sortcolor}{rgb}{0.1, 0.5, 0.1}      % green

\def\lstlanguagefiles{lstlean.tex}

\usepackage[english]{babel}

% definitions for the first chapter, notation from the NWPT'16 abstract
\newcommand\tuple[1]{\langle#1\rangle}

\newcommand\ilSem[2]{{\mathcal {IL}}\left\llbracket#1\right\rrbracket_{#2}}
\newcommand\sem[1]{\left\llbracket#1\right\rrbracket}
\newcommand\tSem[2]{{\mathcal {T}}\left\llbracket#1\right\rrbracket_{#2}}
\newcommand\cSem[2]{{\mathcal C}\left\llbracket#1\right\rrbracket_{#2}}
\newcommand\gSem[1]{\left\llbracket#1\right\rrbracket}
\newcommand\tysem[1]{\left\llbracket#1\right\rrbracket}
\newcommand\compileExp[2] {{\tau_{\textrm{e}}}\left\llbracket#1\right\rrbracket_{#2}}
\newcommand\compileContr[2]{{\tau_{\textrm{c}}}\left\llbracket#1\right\rrbracket_{#2}}
\newcommand\compileOp[2] {{\tau_{\textrm{op}}}\left\llbracket#1\right\rrbracket_{#2}}
\newcommand\eSem[2]{{\mathcal E}\left\llbracket#1\right\rrbracket_{#2}}
\newcommand\cRed[2]{\stackrel{#2}{\Longrightarrow_{#1}}}
\newcommand\opsem[1]{\left\llbracket#1\right\rrbracket}

\newcommand\type[1]{\mathsf{#1}}
\newcommand\obs{\mathbf{obs}}
\newcommand\acc{\mathbf{acc}}
\newcommand\advEnv[2]{#1/#2}

\newcommand\cutPayoff[1]{\mathsf{cutPayoff}(#1)}
\newcommand\zero{\emptyset}
\newcommand\scale{\times}
\newcommand\both{\mathop{\&}}
\newcommand\transl{\uparrow}
\newcommand\promote{\Uparrow}
\newcommand\transfer[3]{{#1(#2 \to #3)}}
\newcommand\letc[3]{\letA\;#1 = #2\;\letB\;#3}
\newcommand\letA{\mathbf{let}}
\newcommand\letB{\mathbf{in}}
\newcommand\ifthen[3]{\mathtt{if} \;#1\;  \mathtt{then} \;#2\;
\mathtt{else} \; #3}
\newcommand\ifwithin[4]{
\mathtt{if} \;#1\; \mathtt{within} \;#2\; \mathtt{then} \;#3\;
\mathtt{else} \; #4}
\newcommand\pto{\rightharpoonup}

\newcommand\id[1]{\ensuremath{\mathit{#1}}}

\reservestyle{\command}{\mathtt}
\command{if[if],model[model],not[not],neg[neg],loopif[loopif],
  payoff[payoff],add[add],sub[sub],mult[mult],div[div],lt[lt],and[and],true[true],
  or[or],scale[scale],translate[translate],zero[zero],
  transfer[transfer],both[both],obs[obs],now[now],let[let],Tnum[Tnum],Tvar[Tvar],
  checkWithin[checkWithin],rNum[rNum],bNum[bNum],Tplus[Tplus],Texpr[Texpr],TnumZ[TnumZ],cond[cond],tplus[tplus]}

\newcommand{\kt}[1]{\texttt{#1}}

%% \theoremstyle{plain}
%% \newtheorem{thm}{Theorem}[section]
%% \newtheorem{conj}{Conjecture}[section]
%% \newtheorem{lemma}{Lemma}[section]

%% \theoremstyle{definition}
%% \newtheorem{defn}{Definition}[section]
%% \newtheorem{remark}{Remark}[section]
%% \newtheoremstyle{case}{}{}{}{}{}{:}{ }{}
%% \theoremstyle{case}
%% \newtheorem{case}{Case}
%% \makeatletter
%% \@addtoreset{case}{theorem}
%% \@addtoreset{case}{lemma}
%% \makeatother
%% \newtheorem{example}{Example}[chapter]

%% \renewcommand\qedsymbol{$\blacksquare$}


% inlined source code
\newcommand{\icode}[1]{\lstinline[mathescape]!#1!}
% inlined source code in theorems
\newcommand{\icodet}[1]{\textup{\lstinline{#1}}}

%% defs for the chapter 2
\newcommand{\Atom}{\mathbb{A}}
\newcommand{\Perm}{\id{Perm}~\Atom}
\newcommand{\SubPerm}[1]{\id{Perm}_{#1}~\Atom}
\newcommand{\swap}[2]{(#1 ~ #2)}
\renewcommand{\action}[2]{#1 \cdot #2}
\newcommand{\aaction}[3]{#2 \cdot_{#1} #3}
\newcommand{\supp}[1]{\id{supp}~ #1}
\newcommand{\asupp}[2]{\id{supp}_{#1}~#2}
\newcommand{\card}[1]{\id{card}~#1}
\newcommand{\Nom}{\mathbf{Nom}}

\newcommand{\finmap}{\stackrel{\textrm{fin}}{\rightarrow}}
\newcommand{\TE}{\id{TE}\,}
\newcommand{\VE}{\id{VE}\,}
\newcommand{\ME}{\id{ME}\,}
\newcommand{\MTE}{\id{MTE}\,}
\newcommand{\SE}{\id{SE}\,}
\newcommand{\Env}{{\textrm{Env}}}
\newcommand{\TEnv}{{\textrm{TEnv}}}
\newcommand{\VEnv}{{\textrm{VEnv}}}
\newcommand{\MEnv}{{\textrm{MEnv}}}
\newcommand{\MTEnv}{{\textrm{MTEnv}}}
\newcommand{\Mid}{{\textrm{Mid}}}
\newcommand{\MTid}{{\textrm{MTid}}}
\newcommand{\Mod}{{\textrm{Mod}}}
\newcommand{\FunSig}{{\textrm{FunSig}}}
\newcommand{\TSet}{{\textrm{TSet}}}
\newcommand{\LSet}{{\textrm{LSet}}}
\newcommand{\NSet}{{\textrm{NSet}}}
\newcommand{\Finset}{{\textrm{Fin}}}
\newcommand{\MTy}{{\textrm{MTy}}}
\newcommand{\kw}[1]{\mbox{\bfseries{#1}}}
\newcommand{\sembox}[1]{\hfill \normalfont \mbox{\fbox{\(#1\)}}}
\newcommand{\titlesembox}[2]{\subsubsection*{#1}\sembox{#2}}
\newcommand{\Dom}{\textrm{Dom}}

\newcommand{\hblue}[1]{\textcolor{blue}{#1}}
\newcommand{\hgreen}[1]{\textcolor{sortcolor}{#1}}
\newcommand{\hviolet}[1]{\textcolor{dkviolet}{#1}}
\newcommand{\hred}[1]{\textcolor{red}{#1}}

\lstset{language=coq}

\title{Nominal Techniques in Coq}

\author{Danil Annenkov}
\institute{University of Copenhagen}
\date{November 16, 2017}

\begin{document}
\maketitle

\AtBeginSection[]
{
   \begin{frame}
       \frametitle{Outline}
       \tableofcontents[currentsection]
   \end{frame}
}

\begin{frame}
  \frametitle{Background}

  \only<1> {\begin{exampleblock}{}
    \large{There are only two hard things in Computer Science: cache
    invalidation and naming things.}

    \vskip5mm
  \hspace*\fill{\small --- Phil Karlton}
  \end{exampleblock}}

  \only<2> {\begin{exampleblock}{}
    \large{There are only two hard things in Computer Science: cache
      invalidation and \textbf{naming things}.}

    \vskip5mm
    \hspace*\fill{\small --- Phil Karlton}
  \end{exampleblock}

  The other side of naming things is to be independent of names!}

\end{frame}

\begin{frame}
  \frametitle{Variable binding}
  \begin{itemize}
  \item Variable binding is ubiquitous concept in the programming language research.
  \item One wants definitions to be independent of the choice of names
    for bound variables.
  \item Easy to deal with in pen-and-paper formalisation.
  \item Notoriously hard to deal with in proof assistants.
  \end{itemize}
\end{frame}

\begin{frame}[fragile]
  \frametitle{Variable binding: Examples}
  \begin{lstlisting}
Definition plusTwo (a : nat) : nat :=
    let b := 2 in a + b.
  \end{lstlisting}
  \pause
  We can pick other names for \icode{a} and \icode{b}:
  \begin{lstlisting}
Definition plusTwo (c : nat) : nat :=
    let d := 2 in c + d.
  \end{lstlisting}
  \pause
  But not arbitrary names: \textbf{variable capture}!
  \begin{lstlisting}
Definition plusTwo (b : nat) : nat :=
    let b := 2 in b + b.
  \end{lstlisting}
\end{frame}

\begin{frame}
  \frametitle{Simply-Typed Lambda Calculus}
  \begin{itemize}
  \item From now on we will switch to the Simply-Typed Lambda Calculus (STLC).
  \item Well-studied, simple binding structure.
  \item The grammar of (raw) lambda terms:\\
    $e \in \texttt{Lam}  ::= v ~|~ \lambda x.e ~|~ e_1 e_2$:
  \end{itemize}
\end{frame}

\begin{frame}
  \frametitle{Variable Convention}
  \begin{exampleblock}{Barendregt's Variable Convention}
     \large{If $M_1,\dots, M_n$ occur in a certain mathematical context
     (e.g. definition, proof), then in these terms all bound
     variables are chosen to be different from the free variables.}
  \end{exampleblock}
\end{frame}

\begin{frame}
  \frametitle{Variable Convention}
  \begin{itemize}
  \item Consider the following typing rule for lambda abstraction:
    \[ \inferrule{\Gamma, x:\tau_1 \vdash e : \tau_2 \\ x \notin \id{dom}(\Gamma)}
              {\Gamma \vdash \lambda x.e : \tau_1 -> \tau_2} \]
  \item In order prove weakening
    \[\forall \Gamma, \Gamma', e, \tau,~~\Gamma |- e : \tau
    \land \Gamma' \subseteq \Gamma => \Gamma' |- e : \tau\]
  in case of lambda abstraction one have to show
  $x \notin \id{dom}(\Gamma')$, knowing only $x \notin \id{dom}(\Gamma)$.
\item This is possible using the variable convention.
\item Not enough to formalise in a proof assistant!
  \end{itemize}
\end{frame}

\begin{frame}
  \frametitle{$\alpha$-equivalence}
  \begin{itemize}
  \item We want to identify lambda-expressions up to renaming
    of bound variables.
  \item For example, $\lambda x.x =_\alpha \lambda y.y$.
  \item We could use substitution. Since $y[x/y] = x$, we say that
    $\lambda x.x =_\alpha \lambda y.y$.
  \item But substitution must be capture-avoiding, otherwise we would identify
    $\lambda x.(x~x)$ and $\lambda y.(y~x)$.
  \end{itemize}
\end{frame}

\begin{frame}
  \frametitle{$\alpha$-conversion with transpositions}
  \begin{itemize}
  \item The \emph{transposition} swaps two names:
    \[ \swap{a}{b}~c =
    \begin{cases}
      a, \text{if } b = c\\
      b, \text{if } a = c\\
      c, \text{otherwise}
    \end{cases} \]
  \item We apply transpositions to \textbf{all} occurrences of
    variables in the lambda-expression:
    $\action{\swap{y}{x}}{(\lambda y.y)} = \lambda x.x$
  \end{itemize}
\end{frame}

\begin{frame}
  \frametitle{$\alpha$-conversion with transpositions}
  Important differences with substitution-based definitions:
  \begin{itemize}
  \item Transpositions cannot lead to variable capture:
    $\action{\swap{y}{x}}{(\lambda y.y~x)} = \lambda x.x~y$
  \item We can implement the capture-avoiding substitution behavior
    using restrictions on variables (we write $x \# y$ for $x \neq y$
    and say ``$x$ is fresh for ``y''):\\
    \smallskip
    pick $z \# x$ and $z \# y$, then
    $\action{\swap{z}{y}}{(\lambda y.y~x)} = \lambda z.z~y$
  \end{itemize}
\end{frame}

\begin{frame}
  \frametitle{Nominal Sets}
  \begin{itemize}
  \item Transpositions are special case of finitary permutations: bijective functions
    on atoms, which are not an identity on a finite subset.
    \pause
  \item Theory of nominal sets [Gabbay and Pitts 1999, Pitts 2013] is a
    \begin{quote}
    mathematical theory of names: scope, binding, freshness.
    \end{quote}
    \pause
    \item Uniform theory based on notions of permutation of variables and finite support.
    \item Applies to various binding structures.
    \item Allows to bring formalisations in a proof assistant closer to pen-and-paper proofs.
    \end{itemize}
\end{frame}

\begin{frame}
  \frametitle{Nominal Sets : Definitions}
  \begin{itemize}
  \item Assume countably infinite set $\{a,b,c,\dots \}$ of \emph{atoms} $\Atom$, that is
    \[ AtomInf : \forall X \in \mathcal{P}_{fin}(\Atom),~\exists a,~a \notin X\]
  \item The \emph{action} of a permutation $\pi$ on a set $X$ as an operation
    $\action \pi -:  X -> X$ with the following properties:
    \begin{itemize}
    \item for any $x\in X$, $\action{\id{id}} x = x$
    \item for any $x\in X$, permutations $\pi_1$ and $\pi_2$, $\action{\pi_1}{(\action{\pi_2}{x})} = \action{(\pi_1 \circ \pi_2)}x$
    \end{itemize}
  \item Finite support in terms of transpositions:
    \[ \forall a,b \notin \supp x.~\action{\swap{a}{b}} x = x \]
  \item A \emph{nominal set} \textbf{X} is a set $X$, equipped with the action
    $\action \pi -$, s.t. each element in $X$ is finitely supported.
  \item We say that $a$ is fresh for $x \in X$ if it is not in the support of $x$
   \[ a \# x = a \notin \supp x\]
  \end{itemize}
\end{frame}

\begin{frame}
  \frametitle{Nominal Set of Lambda Expressions}
  \begin{itemize}
  \item Action (a permutation is applied to \textbf{all} occurrences of atoms uniformly):
    \begin{align*}
    \action \pi v &= \pi v\\
    \action \pi {(\lambda x.e)} &= \lambda (\pi x). \action \pi e\\
    \action \pi {(e_1 e_2)} &= (\action \pi {e_1}) (\action \pi {e_2})
    \end{align*}
  \item Support is a set of \textbf{all} atoms:
    \begin{align*}
    \supp v &= \{v\}\\
    \supp {(\lambda x.e)} &= \{x\} \cup \supp e\\
    \supp (e_1 e_2) & = (\supp {e_1}) \cup (\supp {e_2})
  \end{align*}
  \end{itemize}
\end{frame}

\begin{frame}
  \frametitle{Nominal Techniques: $\alpha$-equivalence}
  We can define $\alpha$-equivalence just in terms of the freshness relation
  and transpositions
  \begin{mathpar}
    \inferrule{\empty}
              {a =_\alpha a}

   \and

    \inferrule{t_1 =_\alpha t_1' \\
               t_2 =_\alpha t_2'}
              {t_1 t_2 =_\alpha t_1' t_2'}
   \and

   \inferrule{\action{\swap{a_1}{b}}{t_1} =_\alpha \action{\swap{a_2}{b}}{t_2} \\
               b \# (a_1,a_2,fv(t_1),fv(t_2))}
              {\lambda a_1 . t_1 =_\alpha \lambda a_2 . t_2}
  \end{mathpar}
\end{frame}

\begin{frame}
  \frametitle{Nominal Techniques: $\alpha$-equivalence}
  \only<1>{\begin{mathpar}
      \inferrule{ }
                {\lambda x.x =_\alpha \lambda y.y}
  \end{mathpar}}
  \onslide<2->{We get $z$ from the $AtomInf$ axiom with $\{x,y\}$}
  \only<2>{\begin{mathpar}
      \inferrule{z \# (x,y,\{x\},\{y\})}
                {\lambda x.x =_\alpha \lambda y.y}
  \end{mathpar}}
  \only<3>{\begin{mathpar}
      \inferrule{\action{\swap{x}{z}}{x}=_\alpha \action{\swap{y}{z}}{y} \\ z \# (x,y,\{x\},\{y\})}
                {\lambda x.x =_\alpha \lambda y.y}
  \end{mathpar}}
  \only<4>{\begin{mathpar}
      \inferrule{\inferrule{ }{z=_\alpha z} \\ z \# (x,y,\{x\},\{y\})}
                {\lambda x.x =_\alpha \lambda y.y}
    \end{mathpar}}
\end{frame}

\begin{frame}
  \frametitle{Nominal Techniques: Summary}
  \begin{itemize}
  \item Actions are very ``uniform'': permutations apply to
    variables in any positions - binding, bound, free.
    \pause
  \item Nice structuring principle for developing formalisations.
    \pause
  \item A simple characterisation of $\alpha$-equivalence.
    \pause
  \item $\alpha$-equivalence can be generalized to various structures
    involving bound variables.
    \pause
  \item The support $t \in Lam/=_\alpha$ is a set of free variables of $t$.
    \pause
  \item A stronger induction principle incorporating the idea
    of bound name independence \lbrack Urban et al. 2007 \rbrack.
  \end{itemize}
\end{frame}

\begin{frame}
  \frametitle{Nominal Techniques: Proof Assistants}
  \begin{itemize}
  \item The most developed library for nominal techniques is the Nominal
    Isabelle package (Isabelle/HOL proof assistant).
  \item A considerable part of the theory of nominal sets can be
    developed in the constructive setting of type theory [Choudhury 2015].
  \item Aydemir, Bohannon, Weirich. Nominal Reasoning Techniques in Coq. 2007.

    \small{\url{http://www.seas.upenn.edu/~sweirich/papers/nominal-coq/}}
  \item Nominal techniques in Coq are part of the DeepSpec summer school course:

    \small\url{https://github.com/DeepSpec/dsss17/tree/master/Stlc}
  \item We apply nominal techniques in the formalisation of a
    higher-order module system.
  \end{itemize}
\end{frame}

\begin{frame}
  \frametitle{Nominal Techniques in Coq}
  \begin{itemize}
  \item Our implementation with examples in lambda-calculus:

    \url{https://github.com/annenkov/stlcnorm}
  \item We use type classes to define permutations and nominal sets.
  \item Type classes provide a convenient way of overloading of main
    notions for various instances of nominal sets.
  \item Proofs of properties are well-suited for automation.
  \item We have developed general definitions: permutations,
    transpositions, nominal sets, freshness relation and so on.
  \item We have implement machinery for generating sets of fresh names.
  \item Our lambda-calculus example contains a definition of $\alpha-equivalence$
    and proofs of equivariance of the typing relation.
  \end{itemize}
\end{frame}

\begin{frame}[fragile]
  \frametitle{Nominal Techniques in Coq: Technical Details}
  Two ways of defining a permutation:
  \begin{itemize}
    \item
    \begin{lstlisting}
  Record Perm :=
      { perm : Atom -> Atom;
        is_biject_perm :  (is_inj perm) /\ (is_surj perm);
        has_fin_supp_perm : has_fin_supp perm}.
    \end{lstlisting}
    Cannot construct an inverse permutation!
  \item
  \begin{lstlisting}
  Record Perm :=
      { perm : Atom -> Atom;
        perm_inv : Atom -> Atom;
        l_inv : (perm_inv ∘ perm) = id ;
        r_inv : (perm ∘ perm_inv) = id ;
        fin_supp : has_fin_supp perm}.
  \end{lstlisting}
  \end{itemize}
\end{frame}

\begin{frame}[fragile]
  \frametitle{Nominal Techniques in Coq: Technical Details}
  The definition of $\alpha-equivalence$:
  \begin{lstlisting}
Inductive ae_exp : NomExp -> NomExp -> Prop :=
| ae_var : forall (a : NomAtom),
    (Var a) =$\alpha$ (Var a)
| ae_lam : forall (a b c : NomAtom) (e1 e2 : NomExp),
    c $\#$ (a, b, fv_exp e1, fv_exp e2) ->
    ((swap a c) @ e1) =$\alpha$ ((swap b c) @ e2) ->
    (Lam a e1) =$\alpha$ (Lam b e2)
| ae_app : forall e1 e2 e1' e2',
    e1 =$\alpha$ e1' ->
    e2 =$\alpha$ e2' ->
    (App e1 e2) =$\alpha$ (App e1' e2')
where "e1 =$\alpha$ e2" := (ae_exp e1 e2).
  \end{lstlisting}
\end{frame}

\begin{frame}
  \frametitle{Nominal Techniques in Coq: More}
  \begin{itemize}
  \item We have applied nominal techniques to a module system formalisation.
  \item We have defined $\alpha-equivalence$ for mutually inductive data structures
    with sets of variables in binding positions.
  \item Ideally, we would like to have a ``nominal'' induction principle.
  \item This requires quotienting with $\alpha-equivalence$.
  \item Defining quotients in Coq is not easy.
  \item Implementation of Aydemir et al. axiomatises a nominal
    induction principle for lambda expressions quotiented with
    $\alpha$-equivalence and provides the soundness proof.
  \item Higher inductive types could be an interesting option.
  \end{itemize}
\end{frame}

\begin{frame}
  \frametitle{Thank you!}
  \Large Thank you for your attention!
\end{frame}
\end{document}


\end{document}
